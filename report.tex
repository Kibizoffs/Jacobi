\documentclass[a4paper]{report}

\usepackage[T2A]{fontenc}
\usepackage[utf8]{inputenc}
\usepackage[russian]{babel}

\usepackage{amsmath}
\usepackage{geometry}
\usepackage{tocbibind} % Добавляем пакет для автоматического включения в оглавление

\geometry{top=2cm, bottom=2cm, left=2.5cm, right=2.5cm}

\title{
\textbf{Московский Государственный Университет имени М.В.\ Ломоносова}\\
\textbf{Факультет вычислительной математики и кибернетики}\\
\textbf{Введение в численные методы}\\
Отчёт по практическому заданию
}

\author{
Студент Кибизов Кирилл, группа 207
}
\date{\number\year}

\begin{document}

\maketitle

\tableofcontents

\chapter*{Постановка задачи}

Было предложено уравнение в частных производных:
\[
k_x \frac{\partial^2 u}{\partial x^2} + k_y \frac{\partial^2 u}{\partial y^2} = 0, \quad (x, y) \in [0, 1] \times [0, 1],
\]
с граничными условиями:
\[
\begin{cases}
u(x, 0) = 0, & x \in [0, 1], \\
u(0, y) = 0, & y \in [0, 1], \\
u(x, 1) = \sin(\pi x), & x \in [0, 1], \\
u(1, y) = 0, & y \in [0, 1].
\end{cases}
\]
Разностная схема:
\[
\begin{cases}
k_x \dfrac{u_{i+1,j} - 2u_{i,j} + u_{i-1,j}}{h^2} + k_y \dfrac{u_{i,j+1} - 2u_{i,j} + u_{i,j-1}}{h^2} = 0, & i = \overline{1,N-1},\ j = \overline{1,N-1}, \\
u_{i,0} = 0, & i = \overline{0,N}, \\
u_{0,j} = 0, & j = \overline{0,N}, \\
u_{i,N} = 0, & i = \overline{0,N}, \\
u_{N,j} = 0, & j = \overline{0,N}.
\end{cases}
\]
где
\[
u_{i,j} \approx u(x_i, y_j), \quad x_i = \dfrac{i}{N}, \quad y_j = \dfrac{j}{N}, \quad h = \dfrac{1}{N}.
\]
Требуется решить данную СЛАУ с помощью итерационного метода Якоби для $N = 100$, рассматривая следующие случаи:
\begin{enumerate}
    \item $k_x = k_y = 1$,
    \item $k_x = 1$, \quad $k_y = 10^6$.
\end{enumerate}

\chapter*{Описание числовых методов}

\chapter*{Анализ применимости числовых методов}

\chapter*{Реализация числовых методов}

\chapter*{Заключение}

\begin{thebibliography}{1}
\renewcommand{\bibname}{Литература}
\bibitem{Samarskiy} Самарский А.\,А. \textit{Введение в численные методы}. — М.: Наука, 1989. — 416 с.
\end{thebibliography}

\end{document}
